% !TEX encoding = UTF-8 Unicode
\documentclass[12pt,a4paper]{report}

\usepackage[brazil]{babel}
\usepackage[utf8]{inputenc}
\usepackage[T1]{fontenc}
\usepackage{graphicx, subfigure}
\usepackage{indentfirst}
\usepackage{color}

\graphicspath{ {images/} }

\title{Análise de redes de comunicação através de \textit{packet sniffing}}
\author{Alexandre Lucchesi Alencar\\
	09/0104471\\
	alexandre@loopec.com.br
	\and
	Pedro Salum Franco\\
	09/0139232\\
	pedro@loopec.com.br
	\and
	Daniel A. M. Sandoval\\
	09/0109899\\
	daniel@loopec.com.br}
\begin{document}
\maketitle

\begin{abstract}
%resumo do relatório que deve conter objetivo do experimento, resultados chaves, pontos maiores de discussão, resultados mais importantes
\end{abstract}

\tableofcontents

\chapter{Introdução}
%introdução: background ou conceitos teóricos relacionados, descrição do equipamento utilizado, restrições no experimento
A transmissão de informação em redes como a Internet ou LANs se dá através da divisão da informação em pacotes, que são transmitidos nos mais diversos meios - Wi-Fi, Bluetooth, rádio, cabos de pares trançados, par metálico, fibra ótica - para chegar da origem ao seu destino.
Os protocolos de rede nas camadas física, enlace, rede, transporte e aplicação são responsáveis por tornar essa comunicação transparente e viável por todo o globo terrestre.

O presente projeto tem como objetivo a análise de redes de comunicação através da técnica conhecida como \textit{packet sniffing}, ou seja, examinar os pacotes que são enviados e recebidos para análise da eficiência da rede de comunicação sendo utilizada.

\section{Fundamentação Teórica}

\paragraph{\textit{Packet sniffing}} Técnica que consiste na análise dos pacotes que trafegam na rede, sejam eles endereçados à estação que está monitorando ou não. Através dessa técnica é possível medir a eficiência e taxa de ocupação de uma rede, além de interceptar toda o conteúdo de comunicação não criptografada.

\paragraph{Roteador} Dispositivo capaz de interligar duas redes realizando tradução de endereços, permitindo a criação de redes cada vez maiores.

\paragraph{\textit{Hops}} Os pacotes transmitidos podem trafegar entre diversas redes para chegar ao seu destino. Quando o pacote passa de uma rede para outra através de um roteador, chamamos isso de \textit{hop}.

\section{Equipamentos Utilizados}

Para atingir os objetivos desse projeto, utilizamos os seguintes equipamentos:

\begin{description}
\item[MacBook Air] Como estação de \textit{packet sniffing}, utilizamos um MacBook Air de 13'' com 4GB de memória RAM e processador Intel Core i7 1.8GHz;
\item[Wireshark] Para poder capturar os pacotes, utilizamos o software Wireshark, que é \textit{open-source} e funciona monitorando atividade na interface de rede e capturando todos os pacotes que chegam a ela;
\item[AirPort Express] Para a criação da rede à qual foi conectado o MacBook, foi utilizado um AirPort Express configurado para criar uma rede WiFi no padrão \(802.11g\), a uma taxa de \(54Mbps\);
\item[D-Link DI-634M] Roteador utilizado para criação de uma subrede para compartilhamento do IP único de saída.
\end{description}


\chapter{Procedimento}
%Procedimento (descrição das ações realizadas, justificativa no caso de mudança de procedimento)

\chapter{Resultados e análise}

\chapter{Conclusão}

\end{document}